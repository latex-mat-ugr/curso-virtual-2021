% Mi primer documento de LaTeX
% Los comentarios se indican con el símbolo de porcentaje "%".
% LaTeX ignorará desde dicho símbolo hasta el final de la línea.

% Declaramos la clase de documento
\documentclass[a4paper, 12pt, twoside]{article} % article, report, book, letter

% Indicamos la codificación de caracteres en que esta escrito este fichero.
% Esto dependerá del editor que estamos usando y podrá configurarse en las
% opciones del editor. 
% Es aconsejable usar la codificación utf8 (es la codificación por defecto 
% en Mac, Linux y en sistemas Windows modernos)
\usepackage[utf8]{inputenc}

% Declaramos el idioma del documento. Esto es importante tanto para poder 
% escribir los signos de puntuación ¡, ¿, º, ª, así como los acentos á, é
% í, ó, ú, ü y la letra ñ. Además, el patrón de división de palabras usado
% al final de una línea se adaptará a dicho idioma.
\usepackage[spanish]{babel}

% Cargamos un paquete para cambiar la tipografía.
% Ver más opciones en https://tug.org/FontCatalogue/
\usepackage[sc]{mathpazo}
\linespread{1.05}         % Palladio needs more leading (space between lines)
\usepackage[T1]{fontenc}

% Cargamos los paquetes de la AMS
\usepackage{amsmath,amssymb,amsthm}

% Indicamos el autor y el título del documento
\title{Mi primer documento en \LaTeX}
\author{Francisco Torralbo}
\date{\today}

% Cuerpo de documento
\begin{document}

\maketitle

\section{Extracto del primer capítulo del Quijote}

\subsection*{Que trata de la condición y ejercicio del famoso hidalgo D. Quijote de la Mancha}

\textsc{En un lugar de la Mancha}, de cuyo nombre no quiero acordarme, no ha mucho tiempo que vivía un hidalgo de los de lanza en astillero, adarga antigua, rocín flaco y galgo corredor. Una olla de algo más vaca que carnero, salpicón las más noches, duelos y quebrantos los sábados, lentejas los viernes, algún palomino de añadidura los domingos, consumían las tres partes de su hacienda. El resto della concluían sayo de velarte, calzas de velludo para las fiestas con sus pantuflos de lo mismo, los días de entre semana se honraba con su vellori de lo más fino. Tenía en su casa una ama que pasaba de los cuarenta, y una sobrina que no llegaba a los veinte, y un mozo de campo y plaza, que así ensillaba el rocín como tomaba la podadera. Frisaba la edad de nuestro hidalgo con los cincuenta años, era de complexión recia, seco de carnes, enjuto de rostro; gran madrugador y amigo de la caza. Quieren decir que tenía el sobrenombre de Quijada o Quesada (que en esto hay alguna diferencia en los autores que deste caso escriben), aunque por conjeturas verosímiles se deja entender que se llama Quijana; pero esto importa poco a nuestro cuento; basta que en la narración dél no se salga un punto de la verdad.

Es, pues, de saber, que este sobredicho hidalgo, los ratos que estaba ocioso (que eran los más del año) se daba a leer libros de caballerías con tanta afición y gusto, que olvidó casi de todo punto el ejercicio de la caza, y aun la administración de su hacienda; y llegó a tanto su curiosidad y desatino en esto, que vendió muchas hanegas de tierra de sembradura, para comprar libros de caballerías en que leer; y así llevó a su casa todos cuantos pudo haber dellos; y de todos ningunos le parecían tan bien como los que compuso el famoso Feliciano de Silva: porque la claridad de su prosa, y aquellas intrincadas razones suyas, le parecían de perlas; y más cuando llegaba a leer aquellos requiebros y cartas de desafío, donde en muchas partes hallaba escrito: 
\begin{quote}
  \itshape
la razón de la sinrazón que a mi razón se hace, de tal manera mi razón enflaquece, que con razón me quejo de la vuestra fermosura, y también cuando leía: los altos cielos que de vuestra divinidad divinamente con las estrellas se fortifican, y os hacen merecedora del merecimiento que merece la vuestra grandeza. 
\end{quote}
Con estas y semejantes razones perdía el pobre caballero el juicio, y desvelábase por entenderlas, y desentrañarles el sentido, que no se lo sacara, ni las entendiera el mismo Aristóteles, si resucitara para sólo ello. No estaba muy bien con las heridas que don Belianis daba y recibía, porque se imaginaba que por grandes maestros que le hubiesen curado, no dejaría de tener el rostro y todo el cuerpo lleno de cicatrices y señales; pero con todo alababa en su autor aquel acabar su libro con la promesa de aquella inacabable aventura, y muchas veces le vino deseo de tomar la pluma, y darle fin al pie de la letra como allí se promete; y sin duda alguna lo hiciera, y aun saliera con ello, si otros mayores y continuos pensamientos no se lo estorbaran.

\begin{flushright}
  Extracto del primer capítulo del Quijote~\cite{cervantesQuijote}\\
\textbf{Miguel de Cervantes}
\end{flushright}


\subsection{Obra de Cervantes}

\begin{itemize}
  \item La Galatea (1585)
  \item El ingenioso hidalgo don Quijote de la Mancha (1605)
  \item Novelas ejemplares (1613)
  \item El ingenioso caballero don Quijote de la Mancha (1615)
  \item Los trabajos de Persiles y Sigismunda (1617)
\end{itemize}

\begin{enumerate}
  \item La Galatea (1585)
  \item El ingenioso hidalgo don Quijote de la Mancha (1605)
  \item Novelas ejemplares (1613)
  \item El ingenioso caballero don Quijote de la Mancha (1615)
  \item Los trabajos de Persiles y Sigismunda (1617)
\end{enumerate}

\begin{description}
  \item[1585] La Galatea 
  \item[1605] El ingenioso hidalgo don Quijote de la Mancha 
  \item[1613] Novelas ejemplares 
  \item[1615] El ingenioso caballero don Quijote de la Mancha 
  \item[1617] Los trabajos de Persiles y Sigismunda 
\end{description}
\section{Poema ``\emph{A un olmo seco}''}

\begin{flushleft}
Al olmo viejo, hendido por el rayo \\
y en su mitad podrido,\\
con las lluvias de abril y el sol de mayo\\
algunas hojas verdes le han salido.
\medskip

\emph{Antonio Machado}~\cite{machadoAntologia}
\end{flushleft}

\section{Incluir fórmulas matemáticas}

Existen dos modos de incluir una fórmula matemática: 
\begin{itemize}
  \item En línea con el texto $\cos^2(x) + \sin^2(x) = 1$, encerrando la ecuación entre \$ y \$ o bien \verb+\(+ y \verb+\)+.
  \item En bloque encerrando la ecuación entre \verb+\[+ y \verb+\]+ o usando el entorno \texttt{equation}
    \[
    \cos^2(x) + \sin^2(x) = 1
    \] 
\end{itemize}

% Bibliografía del documento

% Podemos incluir las referencias manualmente con el entorno thebibliography
\begin{thebibliography}{99}
  \bibitem[Cer2015]{cervantesQuijote} Miguel de Cervantes (2015), \emph{Don Quijote de la Mancha}, Real Academia Española y Asociación de Academias de la Lengua Española. Penguin Random House Grupo Editorial. \textsc{isbn} 978-8420412146
  
  \bibitem[Mac2020]{Machado} Machado, M. (2020). Antología poética. Madrid: Anaya.
\end{thebibliography}

% o bien de forma externa mediante un archivo bibtex
\bibliographystyle{alpha} % plain, unsrt, abbr, alpha
\bibliography{referencias.bib}

\end{document}
